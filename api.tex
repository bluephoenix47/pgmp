\section{API}
\label{sec:api}

This system requires the following four primitives. We present the
primitives here so the reader can understand the examples in the next
section, but delay discussion of the implementation until~\ref{sec:implementation}.
\begin{itemize}
  \item \schemee{profile-query-weight}
  \item \schemee{profile-load-data}
  \item \schemee{profile-dump-data}
  \item \schemee{compile-profile}
\end{itemize}

\schemee{compile-profile} is a parameter used to enable profiling.
\schemee{compile-profile} is \schemee{#f} by default and can be set to
\schemee{'source} or \schemee{'block}. When \schemee{compile-profile} is
\schemee{'source}, compiled programs are instrumented to collect source
level profile information that can be used in macros. When
\schemee{compile-profile} is \schemee{'block}, compiled programs are
instrumented to collect block-level profile information that can be used
in the compiler back-end.

\schemee{profile-dump-data} is used to dump any profile information that
has been collected to a file.

\schemee{profile-load-data} is used to load previously dumped data. 

\schemee{profile-query-weight} is used to retrieve the weighted profile
counts that have been loaded from a file via \schemee{profile-load-data}.
\schemee{profile-query-weight} takes a source or syntax object. When
writing Scheme macros, we primarily use syntax objects since the macro
system manipulates Scheme syntax objects, but manually constructing
source objects from a source file and expression position can be useful
when manipulating generated Scheme code that should correspond to some
higher-level source code. 
